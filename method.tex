\section{Method}
\subsection{Literature review}
To map misconceptions in introductory programming a semi-structured 
literature review was performed. A semi-structured literature review is used 
when the studied area is broad and has been conceptualised in many different 
ways \parencite{Snyder2019}. When conducting a semi-structured literature 
review the purpose will be to find significant findings in the area, however 
it is not mandatory, nor perhaps possible, to review all articles relevant 
to the researched subject. Nevertheless, it is still important to keep the 
review transparent by stating which search words and criterias that have 
been 
used when browsing the articles available on the subject 
\parencite{Snyder2019}.
For this literature review of misconceptions the following criterias have 
been used:
\begin{itemize}
\item The word \emph{misconceptions} should be in the title of the article.
\item The article should be focused on \emph{Python} or other languages that 
have similarities to Python. Articles focused on object-oriented languages 
in 
general might also be relevant.
\item The focus in the article should be on introductory programming for 
high 
school or higher education.
\item Articles with the purpose of deciding which programming language is 
the 
most efficient one to use when teaching introductory programming have been 
sifted out because of the irrelevance for the purpose of this study.
\item Articles which revolve around a tool that measures students' knowledge 
instead of common misconceptions have also been sifted out.
\item An iteration with one search word has been stopped when ten articles 
have had the same findings or if ten articles in a row have been irrelevant.
\end{itemize}
\begin{table}[h]
\centering
\begin{tabular}{ll}
\toprule
Database & Search words\\
\midrule
\multirow{3}{4em}{Google scholar} & common misconceptions in intro to 
programming \\
& common misconceptions cs1 \\
& programming misconceptions student mistakes \\
\bottomrule
\end{tabular}
\caption{Databases and search words used in review}
\label{databasesandwords}
\end{table}
When conducting the literature review several databases and search words 
have 
been used. In \cref{databasesandwords} each word used in each database is 
presented.

The misconceptions found in each chosen article are presented in 
\cref{misconceptions}. 


\subsection{Analysis with variation theory}

Each misconception found in earlier research will be analysed through 
variation theory. The analysis has the purpose of understanding how one can 
teach the different aspects revolving around the misconception at hand, in 
order to avoid this misconception in the future. 