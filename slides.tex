\documentclass[ignoreframetext]{beamer}
\input{preamble.tex}

\begin{document}
% this definition needed to be inside the document
\definecolor{light-green}{HTML}{D5F5E3}

\title{%
  Introductory programming through the lens of variation theory%
}
\author{%
  Celina Soori and
  Daniel Bosk\thanks{%
    Oskar Ejderby,
    Leandros Grigoriadis,
    Björn Hickman,
    Yousef Hilal,
    Beata Johansson,
    Mazen Mardini, and
    Yasmine Schüllerqvist
    helped record students' misconceptions during the course.
  }%
}
\institute{%
  KTH EECS TCS
}
%\date{} % Activate to display a given date or no date (if empty),
         % otherwise the current date is printed 

\mode*

\begin{frame}
  \maketitle
\end{frame}

\begin{abstract}
  \input{abstract.tex}
\end{abstract}

\begin{frame}
  \tableofcontents[subsectionstyle=hide,subsubsectionstyle=hide]
\end{frame}

\mode<all>{\input{introduction}}
\mode<all>{\input{background}}

\input{method}

\mode<all>{\input{results-overview.tex}}

\mode<all>{\input{functions-variables.tex}}
\mode<all>{\input{debugging.tex}}

\input{conditionals.tex}
\input{repetitions.tex}
\input{types.tex}
\input{classes.tex}

\input{findings.tex}

\begin{frame}[allowframebreaks]
\printbibliography
\end{frame}

\appendix

%\section{Misconceptions not yet included}
%
%\input{problem-solving.tex}
%\input{tools.tex}
%
%\section{Misconceptions I have noticed that is not mentioned}
%
%\begin{itemize}
%    \item When returning a value from a function, students misses to 
%capture it into a variable, not understanding why they cannot use the 
%returned value later in the programme. 
%    \item A local list in a function will be changed when passed to a 
%function and then changed in that function. This because it is a 
%reference to the list that is sent to the new function, not a copy of 
%the list. 
%\end{itemize}
%
%\section{Random stuff removed from the article}
%
%\begin{itemize}
%    \item There is also a misconception that variables can hold more than 
%one value at a time \parencite{Doukakis2007}. This misconception can 
%relate to several things:
%    \begin{enumerate*}
%      \item the type system, confusing lists with non-container types, not 
%        seeing a list as a type itself;
%      \item the scope of variables, that the same variable identifier can be 
%    used 
%        for different things in different scopes.
%    \end{enumerate*}
%
%    \item Students that know that a return-statement is needed have 
%difficulties returning the right value or variable from a function 
%\parencite{KumarVeerasamy2016}. 
%
%\end{itemize}
%
\end{document}


