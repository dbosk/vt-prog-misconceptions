\section{Theoretical background}

\subsection{Definition of a misconception}

In order to summarise misconceptions found in earlier research it is 
important to define what a misconception is and how the term is used in this 
article. According to \textcite{NCOL} a misconception is where the student 
understand some critical aspects but misunderstand others, also defined as 
\textcquote[p.~1]{KumarVeerasamy2016}{\textins{a} misconception is an erroneous 
belief, which is not true or valid}. This definition is also used by 
\textcite{MisconceptionsSurvey2017}, where they specify it in a programming 
context to include aspects of syntax, concepts, control flow, learned 
constructs and debugging programs. A misconception can also include errors in 
conceptual understanding of programming. As one can see, the definition of 
misconceptions is broad, and will be used in this article to include all 
errors, misunderstanding, difficulties and so forth. 

\subsection{Variation Theory}

Here I'm thinking we should describe what variation theory is so that we can 
use it in the analysis later on.