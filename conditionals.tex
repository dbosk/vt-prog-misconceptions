\subsection{Conditionals}

A common control structure taught in CS1 is if- and else-statements, where 
the students learn how to create easy conditionals that controls the 
progress of the programme. How a programme will understand and execute an if
-statement is something that students have misconceptions about. A severe 
misconception that \textcite{Plass2015Variables} found was that some 
students believe that an if-statement can control if the programme will 
keep on executing or shut down, depending on if the statement is true or 
false. Students believe that an if-statement that is false will terminate 
the programme, even though a quit-statement has not been introduced. 
Another misconception is that when writing an if and else statement, both 
if and else will be executed, even when the if-statement is true \parencite{
MisconceptionsSurvey2017}.

XXX Add analysis on how we can teach conditionals in a way which help the 
students to understand when and how the code below the conditionals will be 
executed.

A syntax error that is common is when students try to chain conditions in 
an if-statement, for example \verb'if x != a or b', where the correct 
statement should be \verb'if x != a or x != b' \parencite{GuoMarkelZhang2020
}. This misconception is believed to originate from the way the statement 
is read out loud as \verb'if x is not equal to a or b', which in 
mathematical terms is the right way to state it. 

XXX Add analysis on how we can help students to grasp the way if statements 
should be stated in code, apposed to how it is stated in mathematics. 