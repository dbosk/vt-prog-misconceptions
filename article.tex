% !TEX TS-program = pdflatex
% !TEX encoding = UTF-8 Unicode

% This is a simple template for a LaTeX document using the "article" class.
% See "book", "report", "letter" for other types of document.

\documentclass[twocolumn]{article}
\usepackage[utf8]{inputenc} % set input encoding (not needed with XeLaTeX)
\usepackage[english]{babel}
\usepackage{graphicx} % support the \includegraphics command and options

\usepackage[strict]{csquotes}
\usepackage[natbib,style=authoryear-comp,maxbibnames=99]{biblatex}
\addbibresource{bibliography.bib}

\usepackage[all]{foreign}

%%% PACKAGES
\usepackage{booktabs} % for much better looking tables
\usepackage{array} % for better arrays (eg matrices) in maths
\usepackage{paralist} % very flexible & customisable lists (eg. enumerate/itemize, etc.)
\usepackage{verbatim} % adds environment for commenting out blocks of text & for better verbatim
\usepackage{subfig} % make it possible to include more than one captioned figure/table in a single float

\usepackage{hyperref}
% These packages are all incorporated in the memoir class to one degree or another...

%%% The "real" document content comes below...

\title{Student misconceptions in programming through the lens of variation 
theory}
\author{Celina Soori and Daniel Bosk (at the moment)}
%\date{} % Activate to display a given date or no date (if empty),
         % otherwise the current date is printed 

\begin{document}
\maketitle
\tableofcontents

\section{Introduction}

We study misconceptions in introductory programming from the perspective of 
variation theory.
We first survey the existing literature on misconceptions in introductory 
programming.
We then analyze these misconceptions using variation theory.

According to variation theory \parencite[Ch.~2]{NCOL}, each educational 
objective can be divided into different \emph{aspects}.
Consider the educational objective \enquote{the student should be able to use 
functions}.
One aspect of this particular educational objective is local and global scope 
of variables.
Another aspect is returning values from a function.
For a student to achieve the educational objective, she must be able to discern 
the different aspects.
Aspects that the student hasn't discerned are critical aspects.
One necessary condition for learning is that the student is introduced to a 
series of patterns of variation in the critical aspects.
Misconceptions occur when a student has failed to discern (at least) one 
critical aspect.

\paragraph{Purpose}

The purpose of this article is to give an overview of the studies that has been conducted on this subject and analyze these with help of variation theory to understand how one is supposed to teach CS1 to help students grasp important concepts in learning programming. Another purpose with this article is to discuss what further studies we need to conduct to understand deeper what students difficulties there are and how the education can be designed to avoid these. 

We want to do this as a base for future work using variation theory 
\parencite{NCOL}.

\section{Method}

I'm thinking we need a method-section to describe how this research has been conducted. Something about the litterature collecting and so on. 

\section{Concepts}

Added this to explain different words we use in the text and how we interpreted them and use them. Maybe we dont need subsections for every word, but I started this way.

\subsection{Variation Theory}

Here I'm thinking we should describe what variation theory is so that we can use it in the analysis later on.

\subsection{Misconception}

Do we need to describe what we mean with misconception? Do we maybe want to use another word? 

\subsection{Computational thinking}

Maybe we want to include something about computational thinking in this article? 

\subsection{Conceptual Change Theory}

An interesting method to unlearn misconceptions and instead learn the correct conception. Mentioned by Qian \& Lehman. 

\section{Important modules in CS1}

This section is divided to reflect the different concepts that are teached during CS1. Each section will describe what students often are meant to learn and understand in that module, which is then followed with a summary of what different studies have found is difficult for students in that module. 

\subsection{Functions and variables}

Functions and variables is often the first area which is taught in 
introductory programming, an area which holds many misconceptions. This 
section will be divided into several categories, which will reflect the 
different areas that students have trouble grasping.  


\subsubsection{Conceptual understanding of variables}
According to 
\textcite{Kohn2017VariableEvaluation,Plass2015Variables,Doukakis2007}, 
sometimes students believe that variables can hold an entire algorithm and 
therefore see a variable as a function (or a mathematical equation). This will 
create problems when a student creates a variable in belief that the variable 
will dynamically change its value when the equation would change its value, or 
be updated
when the variable is used in the program. Another misconception that goes 
hand in hand with the assumption that a 
variable holds an equation and not a single value, is that if in the return 
statement the student returns an equation, the student believes that the 
return 
value will be that equation, not the value that the equation represents 
\parencite{Kohn2017VariableEvaluation}.

From a variation theoretic perspective, we can say several things about this:
\begin{enumerate}
  \item That variables and functions are interconnected and should be 
treated 
    simultaneously (not \enquote{one thing at a time}), to be able to 
contrast 
    them \parencite[\cf][Ch~6, pp~167--168]{NCOL}.
  \item Unlike in mathematics, every line in a piece of program (in an 
    imperative language) constitutes a new state of the program.
    We must teach this to students through a series of patterns, as 
dictated by 
    variation theory.
\end{enumerate}

A pattern that could be used to help students understand how a variable defined 
by a mathematical statement (for example (\mintinline{python}{x = a + b})) is 
interpreted by Python, could be as followed:

\begin{description}
    \item [Contrast] Show the contrast between where x first is defined by 
addition of variables, then by addition of variables where the value of 
the variables is changed after the definition of x. The contrast is 
shown by printing all these variations, where the value of x is 
invariant, but the value of the variables is the variant. 

    \hfill
\begin{minipage}[t]{0.45\columnwidth}
    \begin{minted}{python}
        def example():
            a = 1
            b = 2
            x = a + b

            return x
    \end{minted}
\end{minipage}
\hfill
\begin{minipage}[t]{0.45\columnwidth}
    \begin{minted}[highlightlines={5-6}]{python}
        def example():
            a = 1
            b = 2
            x = a + b
            a = 2
            b = 3

            return x
  \end{minted}
\end{minipage}
\newline

    Now, according to variation theory, only the aspect in focus should 
change.
    Hence, we kept the name \mintinline{python}{example} in both code 
snippets 
    above.
    This implies that we must show the students how we change this function, 
    not create a new function that with a different name with the 
differences.
    (It would be natural to call one \mintinline{python}{example1} and the 
    other \mintinline{python}{example2}, but that should not be done at this 
    stage.)

    \item [Generalisation] In the generalisation pattern we instead let the 
value of \mintinline{python}{x} vary, and keep the critical aspect (the 
definition) invariant. 
    
\hfill
    \begin{minipage}[t]{0.45\columnwidth}
    \begin{minted}{python}
        def example():
            a = 3
            b = 2
            x = a + b

            return x
    \end{minted}
\end{minipage}
\hfill
\begin{minipage}[t]{0.45\columnwidth}
    \begin{minted}[highlightlines={2-3}]{python}
        def example():
            a = 2
            b = 5
            x = a + b

            return x
  \end{minted}
\end{minipage}
\newline

    \item [Fusion] Here we will let the definition and value of x vary, and 
also use the functionality of functions (in order to teach functions 
and variables simultaneously). 

\hfill
\begin{minipage}[t]{0.3\columnwidth}
  \begin{minted}{python}
    def example(a, b):
        x = a + b

        return x
  \end{minted}
\end{minipage}
\hfill
\begin{minipage}[t]{0.3\columnwidth}
  \begin{minted}[highlightlines={2-3, 5-6}]{python}
    def example(a, b):
        a = 1
        b = 2
        x = a + b
        a = 2
        b = 3

        return x
  \end{minted}
\end{minipage}
\hfill
\begin{minipage}[t]{0.3\columnwidth}
  \begin{minted}[highlightlines={7}]{python}
    def example():
        a = 1
        b = 2
        x = a + b
        a = 2
        b = 3
        x = a + b

        return x
  \end{minted}
\end{minipage}
    
\end{description}



\subsubsection{Defining variables and functions}

But it is not only the right side of the variable definition that students 
can 
have misconceptions about. The name of the variable has been misunderstood 
as 
having power of the value which it holds 
\parencite{MisconceptionsSurvey2017,Sleeman1984}. For example, if a student 
names one variable \emph{max} and another variable \emph{min}, the student 
might think that the variables will strictly only hold the maximum value and 
the minimum value throughout the program, even though the code does not 
carry through this rule. 

We can both explain this phenomenon and propose a teaching design using 
variation theory.
Let's start with the explanation.
When teaching the students we always use proper (\ie relevant) variable 
names 
that relate to the purpose of the variable.
If we never show the students any examples where the variable name is not 
related to its purpose (bad variable names), they cannot separate the 
variable 
naming from its purpose.
This, inevitably, leads to the teaching design:
we must show the students that the variable names are independent of their 
purpose. To achieve this we propose this pattern:

\begin{description}
    \item [Contrast] We show a standard example, then we change a variable 
name from a relevant to an irrelevant one. We show that the program 
still works.
    
    \hfill
    \begin{minipage}[t]{0.45\columnwidth}
        \begin{minted}{python}
            def example(values):
                maximum = 0
                for value in values:
                    if value > maximum:
                        maximum = value
    
                return maximum
        \end{minted}
    \end{minipage}
\hfill
    \begin{minipage}[t]{0.45\columnwidth}
        \begin{minted}[highlightlines={2,4-5,7}]{python}
            def example(values):
                x = 0
                for value in values:
                    if i > x:
                        x = i
    
                return x
      \end{minted}
    \end{minipage}
\newline
    
    \item [Generalisation] We can then \emph{generalise} this by showing 
that we can rename the other variables too. We can even show other 
examples where the variable names are disconnected from the purpose.

    
    \hfill
    \begin{minipage}[t]{0.45\columnwidth}
        \begin{minted}[highlightlines={}]{python}
            def example(l):
                maximum = 0
                for i in l:
                    if i > maximum:
                        maximum = i
    
                return maximum
        \end{minted}
    \end{minipage}
\hfill
    \begin{minipage}[t]{0.45\columnwidth}
        \begin{minted}[highlightlines={3-5}]{python}
             def example(values):
                maximum = 0
                for value in values:
                    if value < maximum:
                        maximum = value
    
                return maximum
      \end{minted}
    \end{minipage}
\newline
    \item [Fusion] When done, we can point out that we name variables 
properly for readability (ease of comprehension), by \emph{contrasting} 
the same example with and without relevant variable names. We can 
follow this by \emph{generalisation}, by showing a previously unseen 
example with unrelated variable names and trying to read it.
    
    \hfill
     \begin{minipage}[t]{0.45\columnwidth}
        \begin{minted}{python}
            def example(l):
                x = 0
                for i in l:
                    if i > x:
                        x = i
    
                return x
        \end{minted}
    \end{minipage}
\hfill
    \begin{minipage}[t]{0.45\columnwidth}
        \begin{minted}{python}
             def example(hi):
                y = []
                
                variable = hi.readlines()
    
                for i in variable:
                    x = i.split()
                    object = Class1(x[0], x[1])
                    y.append(object)
                    
                for i in y:
                    if y.property2 > 18:
                        print(f"Hi {y.property1}! 
                        You are an adult.")
        \end{minted}
    \end{minipage}
\hfill
    
\end{description}
\vspace{5pt}
Other misconceptions that students have when defining variables were found 
by \textcite{GuoMarkelZhang2020},

\begin{enumerate}
    \item When using a variable students have the misconception that the 
      \enquote{pronoun} of the variable can be used later in the programme, 
      instead of the name that was used in the first definition of the 
      variable. For example, if a list has been defined as 
      \mintinline{python}{my_list}, the list is later referenced only as 
      \mintinline{python}{list}.

      From a variation theory perspective, this misconception can be 
managed through a similar pattern as described for the misconception 
of variable names' power of variable values. First by \emph{
contrasting} via examples where the name \mintinline{python}{my_list} 
is used throughout the programme and changed mid-programme 
respectively. 
      
    \hfill
     \begin{minipage}[t]{0.45\columnwidth}
        \begin{minted}{python}
            def example():
                maximum = 0
                for value in my_list:
                    if value > maximum:
                        maximum = value
    
                return maximum
                
            my_list = [1, 3, 2, 5, 4]
            maximum = example()
        \end{minted}
    \end{minipage}
\hfill
    \begin{minipage}[t]{0.45\columnwidth}
        \begin{minted}[highlightlines={3}]{python}
            def example():
                maximum = 0
                for value in list:
                    if value > maximum:
                        maximum = value
    
                return maximum

            my_list = [1, 3, 2, 5, 4]
            maximum = example()
        \end{minted}
    \end{minipage}
\hfill
     
      Then, \emph{generalising} by using different variable names, but 
keeping them throughout the programme. 

        \hfill

        \begin{minted}{python}
            def example():
                for value in my_list:
                    if value > maximum:
                        maximum = value
    

            maximum = 4
            my_list = [1, 3, 2, 5, 4]
            example()
        \end{minted}

\hfill

      The pattern ends with a fusion of the two.

        \hfill

        \begin{minted}{python}
            def example():
                for value in list:
                    if value > maximum:
                        maximum = value


            maximum = 4
            my_list = [1, 3, 2, 5, 4]
            example()
        \end{minted}

\hfill
    
    \item Defining a variable by using another variable students use 
      \mintinline{python}{x == y} instead of \mintinline{python}{x = y}, a 
      misconception supposedly originating from how the statement is read 
out 
      loud as \enquote{x equals y}.

      Here we propose the pattern where we first \emph{contrast} the two 
different statements, by printing the output of \mintinline{python}{x 
== y} and \mintinline{python}{x = y} separately. Both \mintinline{
python}{x} and \mintinline{python}{y} can be pre-defined, to avoid 
the programme throwing an error. However, in the contrast we could 
also include the example where an error is thrown, this to \emph{
contrast} even further. 

      \hfill
     \begin{minipage}[t]{0.3\columnwidth}
        \begin{minted}{python}
            def example():
                x = 1
                y = 2
                x == y
    
                return x
        \end{minted}
    \end{minipage}
\hfill
    \begin{minipage}[t]{0.3\columnwidth}
        \begin{minted}[highlightlines={4}]{python}
            def example():
                x = 1
                y = 2
                x = y
                
                return x
        \end{minted}
    \end{minipage}
\hfill
    \begin{minipage}[t]{0.3\columnwidth}
        \begin{minted}[highlightlines={3}]{python}
            def example():
                y = 2
                x == y 
                # Will throw an 
                # error
                return x
        \end{minted}
    \end{minipage}
\hfill
    
      The \emph{generalisation} for this pattern is easy, and consists of 
several variable-definitions where other variables are used in the 
definition. 

      \hfill
     \begin{minipage}[t]{0.45\columnwidth}
        \begin{minted}{python}
            def example(y):
                x = y
    
                return x   
        \end{minted}
    \end{minipage}
\hfill
    \begin{minipage}[t]{0.45\columnwidth}
        \begin{minted}[highlightlines={}]{python}
            def example():
                a = 3
                b = a
    
                return b
        \end{minted}
    \end{minipage}
\hfill
     
      
      The last pattern, \emph{Fusion} should consist of both variable-
definitions, and where the variables also are used in comparison-
statements. 

      \begin{lstlisting}[language=Python]

        def example1():
            XXX Not really sure of a good example for this... 
            maybe an absolut-value example?
                
    \end{lstlisting}
    \item Writing definitions of variables from left-to-right 
      (\mintinline{python}{a+b = c}) instead of right-to-left 
      (\mintinline{python}{c = a+b}). The same can be seen when using 
functions 
      in the definition of variables, for example instead of writing 
      \mintinline{python}{x=parse(input())} the students write 
      \mintinline{python}{parse(x) = input()}.

      This misconception can be seen as a misconception of what the 
functionality of the left and right side of a variable definition is. 
To \emph{contrast} this, we need to create code which will throw 
errors, since it is not possible to do function calls on the left 
side of a variable definition. This can be done with examples from 
both misconceptions mentioned, where we do it the right way and the 
wrong way, using the same examples as \textcite{GuoMarkelZhang2020}.

            \hfill
     \begin{minipage}[t]{0.45\columnwidth}
        \begin{minted}{python}
            def example():
                a = 1
                b = 2
                c = a + b  
        \end{minted}
    \end{minipage}
\hfill
    \begin{minipage}[t]{0.45\columnwidth}
        \begin{minted}[highlightlines={4}]{python}
            def example():
                a = 1
                b = 2
                a + b = c 
                # Will throw an 
                # error  
        \end{minted}
    \end{minipage}
\hfill
\vspace{5pt}
\hfill 
        \begin{minipage}[t]{0.45\columnwidth}
            \begin{minted}[highlightlines={}]{python}
                def example():
                    x = parse(input())
            \end{minted}
        \end{minipage}
        \hfill
        \begin{minipage}[t]{0.45\columnwidth}
            \begin{minted}[highlightlines={2}]{python}
                def example():
                    parse(x) = input() 
                    # Will throw an 
                    # error
            \end{minted}
        \end{minipage}
        \hfill
      
      We then \emph{generalise} it by writing an example, with several 
function calls which will not throw errors. 

        \begin{minted}[highlightlines={}]{python}
            def example():
                number = int(input("Write a number "))
                new_number = number + 1
                string_number = str(abs(new_number))
        \end{minted}

\hfill
      
      In this pattern it will however not be possible to \emph{fuse} the 
invariant and variant with each other, since the latter will throw an 
error for the programme. XXX Or can you think of a fusion Daniel?
\end{enumerate}


Since Python is an interpreted language, the placement of the definition of 
a function is important, something that differs from a compiled language. 
This gives 
room for a misconception for students that have learned to code in for 
example Java, where the definition of a function can be below a call of the 
function. 

XXX Add analysis \textbf{IF} we want to include this misconception. However 
it can be seen as out of the scope of our article, since we focus on novice 
programmers. But we still have students that might have been exposed to 
Java in high-school, so it might still be interesting to include?

\subsubsection{Arguments and return values of functions}

If we move on to the relationship between variables and functions we can see 
more misconceptions that students have; for example where input and output 
arguments 
come from and go to \parencite{Ragonis2005OOP}.
The first difficulty is how students understand and treat return-values, 
where these student misconceptions have been found: 

\begin{enumerate}
    \item Missing to return the variable when a function is supposed to, 
expecting the function to return it by default \parencite{
Kurvinen2016,KumarVeerasamy2016}.

    \item Believe that a print-statement at the end of a function will act 
as a return statement \parencite{MisconceptionsSurvey2017}.

\end{enumerate}
 

These two misconceptions are connected to each other, and what they have in 
common is the trouble to return a value, the \emph{right} value, from a 
function. However, the two misconceptions can be, and according to us 
should be, treated separately. To help the students understand how to write 
a correct return-statement for different functions we propose these two 
patterns for the two misconceptions:

\begin{enumerate}
    \item To help the students understand that a function will not return 
the correct value by default, we start with \emph{contrasting} with the 
help of a function which in the first example does not return a value, 
and in the second return the value. We name the functions \mintinline{
python}{find_maximum}, this to trigger the misconception since this 
might "trick" students into believing that the return value will 
automatically be the expected value (in this case the maximum). 

    \hfill
     \begin{minipage}[t]{0.45\columnwidth}
        \begin{minted}{python}
            def find_maximum(values):
                maximum = 0
    
                for value in values:
                    if value > maximum:
                        maximum = value 


            
            x = find_maximum([1,3,2])
        \end{minted}
    \end{minipage}
\hfill
    \begin{minipage}[t]{0.45\columnwidth}
        \begin{minted}[highlightlines={8}]{python}
            def find_maximum(values):
                maximum = 0
    
                for value in values:
                    if value > maximum:
                        maximum = value
                        
                return maximum
            
            x = find_maximum([1,3,2])
        \end{minted}
    \end{minipage}
\hfill

    Then we \emph{generalise} by showing two different examples where we 
return the expected value. 
     \hfill
        \begin{minted}{python}
            def read_file(filename):
                file = open(filename, "r")
                lines = file.readlines()
    
                return lines
        \end{minted}

\hfill

        \begin{minted}[highlightlines={}]{python}
            def calculate_average(values):
                values_sum = sum(values)
                average = values_sum/len(values)
         
                return average
        \end{minted}

\hfill

    And the last step of the pattern, the \emph{fusion}, consists of an 
example where we use functions that return the expected value, and not.
    \hfill

        \begin{minted}{python}
        def read_file(filename):
            file = open(filename, "r")
            lines = file.readlines()

            return lines
            
        def calculate_average(values):
                values_sum = sum(values)
                average = values_sum/len(values)

        def main():
            lines = readfile("test.txt")
            values = []
            for line in lines:
                values.append(int(line[0]))
            average = calculate_average(values)
            print(average)
        \end{minted}
\hfill

    \item The latter misconception we believe originates from the students' 
misconception that what they see in the terminal is what happening in 
the programme. So when we print the value that we want to be returned, 
the programme will see it and can later use it. We start the pattern 
with \emph{contrasting} this concept, using the same example as above.

        \hfill
     \begin{minipage}[t]{0.45\columnwidth}
        \begin{minted}{python}
            def find_maximum(values):
                maximum = 0
                for value in values:
                    if value > maximum:
                        maximum = value
    
                print(maximum)

            x = find_maximum([1,3,2])
            print(x)
        \end{minted}
    \end{minipage}
\hfill
    \begin{minipage}[t]{0.45\columnwidth}
        \begin{minted}[highlightlines={7}]{python}
            def find_maximum2(alues):
                maximum = 0
                for value in values:
                    if value > maximum:
                        maximum = value
                        
                return maximum
    
            x = find_maximum2([1,3,2])
            print(x)
        \end{minted}
    \end{minipage}
\hfill

    Here the students will see the maximum in the terminal when calling 
\mintinline{python}{find_maximum}, and might therefore believe that x 
will hold the value the function printed. The \emph{generalisation} for 
this misconception will be the same as for the first misconception, but 
the \emph{fusion} will differ slightly. When we \emph{fuse} in this 
pattern, we instead vary between returning the correct value and only 
printing it.
     \hfill
        \begin{minted}{python}
        def read_file(filename):
            file = open(filename, "r")
            lines = file.readlines()

            print(lines)
            
        def calculate_appearences(lines):
            num_appearences = 0
            for line in lines:
                if line[0] == "Adam":
                    num_appearences += 1
            return num_appearences
            

        def main():
            lines = readfile("test.txt")
            appearences = calculate_appearences(lines)
            print(appearences)
        \end{minted}
\hfill
    
\end{enumerate}

A student might also write a function which returns a value, but that value 
is not stored nor being used later in the program 
\parencite{AltadmriBrown2015}.


We draw the conclusion that this misconception can be the product of 
students believing that a function has to have a return-statement to end, 
and that the programme will throw an error if a function misses a return-
statement. We propose this pattern to avoid this misconception:

\begin{description}
    \item[Contrast] To contrast this misconception, we write a function 
that is not meant to return a value, with and without a return-
statement. To contrast it even further we print the return-value 
\mintinline{python}{None}, to show the students that the two functions 
return the same value. 

      \hfill
     \begin{minipage}[t]{0.45\columnwidth}
        \begin{minted}{python}
            def example():
                print("Hello world!")

                return

            print(example())
        \end{minted}
    \end{minipage}
\hfill
    \begin{minipage}[t]{0.45\columnwidth}
        \begin{minted}[highlightlines={4}]{python}
            def example():
                print("Hello world!")



            print(example())
        \end{minted}
    \end{minipage}
\hfill

    
    \item[Generalisation] In the generalisation of this pattern we write 
different functions which all misses a return-statement, to show the 
students that the functions still executes and ends without it. 

    \hfill
     \begin{minipage}[t]{0.45\columnwidth}
        \begin{minted}{python}
            def example(lines):
                file = open("test","w")
                for line in lines: 
                    file.write(line)
                file.close()
        \end{minted}
    \end{minipage}
\hfill
    \begin{minipage}[t]{0.45\columnwidth}
        \begin{minted}[highlightlines={}]{python}
            def example():
                print("Menu options")
                print("A. Open file")
                print("B. Add person")
                print("C. Delete person")
        \end{minted}
    \end{minipage}
\hfill
    \item[Fusion] We know fuse the invariant and the variant in the earlier 
examples, with a programme that have functions with and without return-
statements. 
    \hfill
        \begin{minted}{python}
        def print_menu(filename):
            print("Menu options")
            print("A. Open file")
            print("B. Add person")
            print("C. Delete person")
            
        def add_person(file):
                name = input("Name? ")
                file = open("file","a")
                file.write(name)
                file.close()
                
            return

        def main():
            file = "names.txt"
            print_menu()
            option = input()
            if option == "B":
                add_person(file)
        \end{minted}
\hfill

\end{description}

It is not only the return value that is difficult to grasp for students, the 
input arguments are also a difficult concept for some students. When 
calling a 
function \textcite{AltadmriBrown2015} found that students often have 
trouble passing the right parameter when invoking a function, for example 
by inserting data in a wrong type in the function call. 

This misconception students have when invoking functions, shows a gap in 
the conceptual understanding of function arguments and parameters. To avoid 
this and to help students grasp the relationship between function arguments 
and parameters in function calls we propose this pattern:

\begin{description}
    \item[Contrast]
    \item[Generalisation]
    \item[Fusion]
\end{description}


\Textcite{Fleury1991} researched the 
misconceptions students have when using parameters in functions.
She found that students had constructed their own rules for the using 
of global and local variables which are connected to the use of variables in 
functions. These are the misconceptions she found, 

\begin{enumerate}
    \item when changing a local variable in a function, the variables with 
the same name is changed for the whole program

    \item if the local variable is not an argument in the function-call, 
the program will go back to where the function was called and search 
for it there

    \item if a function references to a global variable that is not a 
function argument, it will create an error in the program

    \item if a global variable is changed in the function body, the new 
value will not be reachable for the rest of the program if not returned 
by the function
\end{enumerate}

XXX Add analysis on how one can teach the difference between local and 
global variables and how they are used in functions. 

\begin{description}
    \item[Contrast]
    \item[Generalisation]
    \item[Fusion]
\end{description}

\subsubsection{Variables in mathematics vs programming}

The difference between variables in programming and variables in 
mathematics is something that some students do not grasp. If a student in a 
variable 
definition uses, on the right side of the equal symbol, a variable that is 
not 
defined, but the variable on the left side is already defined, they think 
that 
the computer will solve this as an equation \parencite{Plass2015Variables}. 
This assumption made by the students was also discovered by 
\textcite{Kohn2017VariableEvaluation} when giving the students the 
definition 
\mintinline{python}{x = x + 1}. If you look at this definition with a 
mathematical 
perspective you will see an unsolvable equation, which is also what some of 
the 
students saw. They did not see that the \mintinline{python}{x} to the left 
is the 
variable, 
and that the \mintinline{python}{x} to the right only holds a value. This 
definition is  
easier to understand for a novice programmer, according to 
\textcite{Kohn2017VariableEvaluation}, when we instead write this 
definition as 
\mintinline{python}{x += 1}. 

XXX Add analysis on how to help students understand the difference between 
variable definitions in programming and equations in mathematics

\begin{description}
    \item[Contrast]
    \item[Generalisation]
    \item[Fusion]
\end{description}




\subsection{Classes and objects}

The concept of classes and objects in object-oriented languages is difficult and a basic understanding of objects is something that many CS1 students lack \parencite{Kaczmarczyk2010}. This is emphasized by \textcite{Ragonis2005OOP} who in their study noticed a number of dire misconceptions, two which are presented below. 

\begin{enumerate}
    \item An instance of a class can be created within the class' method.

    \item It is possible to define a method which does not access any of the class' attributes
\end{enumerate}

XXX Add analysis from variation theory


\parencite{Ragonis2005OOP} also found that students have a hard time visualising the class as a template for a type of object. Instead the students have the image of the class as a collection of objects and that the class' methods have the power to change, add and delete objects that are class-instances. Similar misconceptions has been characterised by \textcite{Holland1997}, where the students believe that \begin{enumerate*}
    \item an object is a variable that can only hold one or several values of the same type and
    \item a class is strictly a data base
\end{enumerate*}. These particular misconceptions can be traced back to the first classes the students write, which are often good substitutes to data bases and therefor shapes these student misconceptions. 

XXX Add analysis one how we can teach classes with easy examples and still manage to avoid the misconception that a class is equal to a data base. Maybe we can contrast it with the already existing objects in python (Strings, lists, integers etc).

The last concept that \textcite{Holland1997} discuss is the concept of storing the objects in the programme. Some students believe that the attributes of an object are the objects identifier, which leads to the misconception that there can not be two objects that have the same attributes, and therefor that one attribute of every object must be unique otherwise the programme will not be able to store it. The concept that every object has its own memory space and are stored separately is hard to grasp for some students \parencite{Holland1997,Ragonis2005OOP}. 

XXX Add analysis in how we can teach how objects are stored with the help of variation theory.


\subsection{Repetitions}

In CS1 students usually learn about repetitions, which includes for- and 
while-loops and in some cases recursion.

Loop constructions can be hard to trace and understand for novice students, 
for instance when a loop starts, ends and what is repeated and not repeated 
in the loop \parencite{Sekiya2013,KumarVeerasamy2016,Kaczmarczyk2010}. This 
was something \textcite{Sleeman1984} also realised when studying high 
school students writing and debugging loop-structures. A common 
misconception that the students had was that if the loop contained a print-
statement, the students thought that the only thing repeated inside the 
loop was the string they saw in the terminal. The difficulties students 
have in tracing the code linearly when entering a loop is according to 
\textcite{KumarVeerasamy2016} because of the lack of understanding the 
students have of the looping technique and the amount of cognitive skills 
the tracing takes.

XXX Add analysis on how we can help students trace loops and understanding 
how the loop-structure works.

When students define and use loops, \textcite{GuoMarkelZhang2020} found 
three common misconceptions that the students had about the loop-statement, 

\begin{enumerate}
    \item A misconception about which variable in a loop defined as 
      \mintinline{python}{for item in items}, that is supposed to be use to 
      extract different information.
    \item The misconception that \mintinline{python}{for i in 100} will iterate 
      a hundred times, even though Python require a specified range.
    \item When defining a while-loop, the misconception is that one can write 
      \mintinline{python}{while i <= 100} without initialising 
      \mintinline{python}{i} beforehand, and that \mintinline{python}{i} will 
      automatically increase with 1 inside the loop.
\end{enumerate}

Another difficult part of the loop technique is to understand how an if-
statement inside a loop is executed. \textcite{Sekiya2013} found in their 
studies that the combination of the two control structures created 
misconceptions. For instance the students thought that the variables in the 
conditional part of the loop-construction was control variables or the 
output from the loop. The students in the studies often got confused and 
started to misplace the different variables that are defined when writing 
an if-statement in a for-loop.

XXX Add analysis on how we can teach the combination of loops and 
conditionals in a way which will avoid misconceptions about the different 
variables used. 
\subsection{Data types}

When introduced to programming students will often encounter several 
different data types, for example strings, integers, arrays and dictionaries. 
As excepted when learning about several data types during a short time-span 
different misconceptions occur. 

When using data types in CS1 the students are often required to use some kind 
of comparison between variables. This creates situations where the students 
try to compare different data types to each other, which shows that the 
students are not completely familiar to the difference between data types and 
how they are initialised and then later used \parencite{Kurvinen2016}. When 
objects are introduced to the students later in the course it will create 
more confusion when the students are suppose to compare objects with simple 
data types, which can be made easier if comparison between types are repeated 
by the teacher at the end of the course \parencite{Kurvinen2016}. XXX Do we 
think this is interesting enough to mention? Is it part of the scope?

XXX Add analysis on how to introduce different data types so that the 
students understand the difference between them


 \textcite{Kurvinen2016} found in their studies that it is not intuitive for 
students that the indexation of an array starts at 0 instead of 1. They also 
noticed that students have a hard time figuring out how to loop through an 
array's elements by using the elements index. The same problem was found by 
\textcite{KumarVeerasamy2016} who found that students often are of by one 
index when looping through an array using index, causing index-error when 
trying to extract an element with an index larger then the length of the 
list. They also saw a tendency to use negative index-numbers when 
unnecessary to extract elements from an array.

 XXX Insert analysis on how we can teach indexation of arrays


\textbf{Fun things to add:}

- "Omitting quotes for strings: In natural language, there is no need to use 
quotes for prose. Omitting a single-word string like foo(Alice) is legal code 
but accesses an undefined variable; omitting a multi-word string like ‘foo(My 
name is Alice)’ leads to a parse error. Escape sequences like \" are also 
confusing since they are not needed in English." \parencite{GuoMarkelZhang2020}

- "When numerical data is read from files or terminal input, they often start 
as strings. If they are not properly converted to numbers, it is still 
possible to use math operators like + and > on them, which will perform 
string concatenation and comparison, respectively. These can lead to subtle 
semantic and logic errors, even though the code does not crash" \parencite{
GuoMarkelZhang2020}
\subsection{Problem solving}

\subsubsection{Role in the syllabus}

\subsubsection{Difficulties that can occur}


Here I want to have some articles about how math-problems will make it harder 
for students to solve the problem. Quote from Veerasamy et al \emph{This 
study analysis also explored that novices of programming struggled in writing 
code for math-related Questions 6 and 7 (refer Table C1). Nearly 66\% of 
students did not do well in the mathematical problem-based questions though 
explained and allowed to surf the Internet to seek for more details during 
the exam hours. A neo-Piagetian theory of cognitive development stated that 
students who are at the concrete operational stage struggle to write large 
programs with partial specifications, although they can write small programs 
from well-defined specifications (Teague et al., 2012).}

Also I would want to include difficulties students have when debugging the 
code 
and trying to find errors. Students often have a problem with tracing the 
code, 
something that is discussed by \textcite[p.~20]{Sleeman1984}. On the same 
subject as above: In what order a program will be executed in, Programming 
misconceptions in an introductory level programming course exam by Einari 
Kurvinen, Niko Hellgren, Erkki Kaila, Mikko-Jussi Laakso, Tapio Salakoski

Would also maybe like to mention how a lab instruction should be to help 
students get the right knowledge from the lab. Is discussed somewhere in 
Yizhou Qian and James Lehmans article I think.


\subsection{Conditionals}

A common control structure taught in CS1 is if- and else-statements, where 
the students learn how to create easy conditionals that controls the 
progress of the programme. How a programme will understand and execute an if
-statement is something that students have misconceptions about. A severe 
misconception that \textcite{Plass2015Variables} found was that some 
students believe that an if-statement can control if the programme will 
keep on executing or shut down, depending on if the statement is true or 
false. Students believe that an if-statement that is false will terminate 
the programme, even though a quit-statement has not been introduced. 
Another misconception is that when writing an if and else statement, both 
if and else will be executed, even when the if-statement is true \parencite{
MisconceptionsSurvey2017}.

XXX Add analysis on how we can teach conditionals in a way which help the 
students to understand when and how the code below the conditionals will be 
executed.

A syntax error that is common is when students try to chain conditions in 
an if-statement, for example \verb'if x != a or b', where the correct 
statement should be \verb'if x != a or x != b' \parencite{GuoMarkelZhang2020
}. This misconception is believed to originate from the way the statement 
is read out loud as \verb'if x is not equal to a or b', which in 
mathematical terms is the right way to state it. 

XXX Add analysis on how we can help students to grasp the way if statements 
should be stated in code, apposed to how it is stated in mathematics. 

\subsection{Things that doesn't fit in anywhere else}

\subsection{Tools}

IDE: What IDE is best for CS1? What difficulties can occur when using 
different IDEs? Should we recommend one? Quote from Qian \& Lehman \emph{
Although many other syntactic-level errors are reported in previous research (
see Altadmri and Brown (2015), Hristova et al. (2003), and Sorva (2012)), we 
do not discuss them in depth here, because problems in syntactic knowledge 
are often easy to detect and fix. Perhaps that is why they are often noted as 
the most frequent mistakes novices make (Altadmri and Brown 2015; Jackson et 
al. 2005). A compiler or a modern integrated development environment (IDE) 
may be able to find them and then provide error messages or hints for 
correction.}




\section{Analysis}

The analyze will be performed with the help of variation theory. 

\section{Discussion}

Here we can discuss what we want to analyze and research further. We might also want to discuss how to design the education to avoid everything above.

\newpage

\begin{thebibliography}{9}

\bibitem{texbook}
Altadmri, A. Brown, N. C. C. (2015). 37 Million Compilations: Investigating Novice Programming Mistakes in Large-Scale Student Data. \emph{SIGCSE '15: Proceedings of the 46th ACM Technical Symposium on Computer Science Education}. February 2015. Pages 522-527.
\url{https://dl.acm.org/doi/abs/10.1145/2676723.2677258}

\bibitem{texbook}
Fleury, A. E. (1991). Parameter passing: The rules the students construct. \emph{SIGCSE’91:  In Proceedings of the 22nd SIGCSE Technical Symposium on Computer Science Education}. Pages 283-286.
\url{http://dx.doi.org/10.1145/107004.107066}

\bibitem{texbook}
Goldman, K. Gross, P. Heeren, C. Herman, G. Kaczmarczyk, L. Loui, M. C. Zilles, C. (2010). Identifying Important and Difficult Concepts in Introductory Computing Courses using a Delphi Process. \emph{SIGCSE '08: Proceedings of the 39th SIGCSE technical symposium on Computer science education} March 2008. Pages 256–260.
\url{https://dl.acm.org/doi/10.1145/1352135.1352226} 

\bibitem{texbook}
Holland, S. Griffiths, R. Woodman, M. (1997). Avoiding Object Misconceptions. \emph{ACM SIGCSE Bulletin} Volume 29, Issue 1, March 1997. Pages 131–134.
\url{https://dl.acm.org/doi/pdf/10.1145/268084.268132} 

\bibitem{texbook}
Kohn, T.  (2017). \emph{Variable Evaluation: an Exploration of Novice Programmers’ Understanding and Common Misconceptions}
\url{https://dl.acm.org/doi/abs/10.1145/3017680.3017724}

\bibitem{texbook}
Kurvinen, E. Hellgren, N. Kaila, E. Laakso, M. J. Salakoski, T. (2016). Programming misconceptions in an introductory level programming course exam. \emph{ITiCSE '16: Proceedings of the 2016 ACM Conference on Innovation and Technology in Computer Science Education}. July 2016. Pages 308–313. 
\url{https://doi.org/10.1145/2899415.2899447}

\bibitem{texbook}
Kumar Veerasamy, A. D’Souza, D. Laakso, M. J. (2016). Identifying Novice Student Programming Misconceptions and Errors From Summative Assessments. \emph{Journal of Educational Technology Systems.} Vol 45, Issue 1.  
\url{https://doi.org/10.1177/0047239515627263}

\bibitem{texbook}
Plass-Oude Bos, D.  (2015). Identifying and Addressing Common Programming Misconceptions with Variables (Part 1). \emph{University of Twente}.
\url{https://essay.utwente.nl/70455/1/Oude\%20Bos\%20Danny\%20-\%20S0021407\%20-\%20Afstudeerscriptie.pdf}

\bibitem{texbook}
Ragonis, N. Ben-Ari, M. (2005). A long-term investigation of the comprehension of OOP concepts by novices. \emph{ISSN: 0899-3408 (Print) 1744-5175 (Online) Journal homepage}. 15:3. 203-221. 
\url{https://doi.org/10.1080/08993400500224310}


\bibitem{texbook}
Sekiya Y. Yamaguchi, K. (2013) Tracing quiz set to identify novices' programming misconceptions. \emph{Koli Calling '13: Proceedings of the 13th Koli Calling International Conference on Computing Education Research} November 2013. Pages 87–95.
\url{https://doi.org/10.1145/2526968.2526978} 

\bibitem{texbook}
Sleeman, D. Putnam, R. T. Baxter, J. A. Kuspa, L. K. (1984). Pascal and High-School Students: A Study of Misconceptions. \emph{Technology Panel Study of Stanford and the Schools}. Occasional report \#009. 
\url{https://files.eric.ed.gov/fulltext/ED258552.pdf}

\bibitem{texbook}
Qian, Y. Lehman, J. (2017). Students’ Misconceptions and Other Difficulties in Introductory Programming: A Literature Review. \emph{ACM Trans. Comput. Educ.} 18, 1, Article 1 (October 2017).
\url{https://doi.org/10.1145/3077618} 



\end{thebibliography}
\printbibliography

\end{document}
