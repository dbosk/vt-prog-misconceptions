\mode*
\section[Misconceptions and teaching]
  {Misconceptions and how to teach introductory programming}
\label{misconceptions}

This section is divided to reflect the different concepts that are taught 
throughout CS1. Each section will describe what students often are meant to 
learn and understand in that module. Something which is followed by a 
summary 
of what different studies have found is difficult for students in that 
particular
module and which common misconceptions that students may have. Each common 
misconception will be analysed through the lens of variation theory, with 
the purpose of explaining how it can be avoided by adjusting the way 
the specific term or concept is being introduced or taught.

Variation theory dictates that we teach using specific patterns of variation.
We should start with a contrast pattern in a critical aspect, followed by a 
generalization pattern for the same critical aspect and finally tie several 
aspects together using a fusion pattern.

Let's get started.

\begin{frame}
  \begin{block}{Areas covered}
    \begin{itemize}
      \item \alert<3>{Functions and variables}
      \item Conditionals
      \item Repetitions (iterations, loops, recursion)
      \item Types
      \item Classes
    \end{itemize}
  \end{block}

  \pause

  \begin{remark}
    \begin{itemize}
      \item We also found misconceptions among our students, not in the 
        literature.
    \end{itemize}
  \end{remark}
\end{frame}
