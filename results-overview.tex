\section{Misconceptions in introductory programming}
\label{misconceptions}

This section is divided to reflect the different concepts that are taught 
throughout CS1. Each section will describe what students often are meant to 
learn and understand in that module. Something which is followed by a 
summary 
of what different studies have found is difficult for students in that 
particular
module and which common misconceptions that students may have. Each common 
misconception will be analysed through the lens of variation theory, with 
the purpose of explaining how it can be avoided by adjusting the way 
the specific term or concept is being introduced or taught.

Let us start with an example to illustrate the outline.
We will use a particular feature related to default arguments in Python that 
few expect, hence most readers will hopefully not know about this and get the 
intended experience.
(So this first set of patterns is not intended for students learning to program 
for the first time, but rather the instructors for whom this paper is 
intended.)

Variation theory dictates that we teach using specific patterns of variation.
We should start with a contrast pattern in a critical aspect, followed by a 
generalization pattern for the same critical aspect and finally tie several 
aspects together using a fusion pattern.

Let's get started.

\begin{description}
  \item[Contrast] The contrast pattern requires two examples to create 
    contrast.
    The left-hand example should be read first and then the right-hand 
    example will highlight the changes made to the left-hand example to get 
    the right-hand one.
    (So that one can make the changes oneself.)

    We assume that the reader is familiar with default values for arguments in 
    Python.
    We define a function \mintinline{python}{expand} that takes a list as an 
    argument and expands it by appending an element \mintinline{python}{1} 
    (left-hand code below).
    Then we run some examples.

    \begin{minipage}[t]{0.45\columnwidth}
      \begin{pyblock}[default1]
def expand(x=[]):
  return x + [1]


print(
  f"[1] -> {expand([1])}\n"
  f"[2] -> {expand([2])}\n"
  f"()  -> {expand()}\n"
  f"()  -> {expand()}\n"
)
      \end{pyblock}
      \vspace{0.5em}
      This yields the output
      \vspace{0.5em}
      \printpythontex[verbatim]
    \end{minipage}
    \hfill
    \begin{minipage}[t]{0.45\columnwidth}
      \begin{pyblock}[default2][highlightlines={2-3}]
def expand(x=[]):
  x.extend([1])
  return x

print(
  f"[1] -> {expand([1])}\n"
  f"[2] -> {expand([2])}\n"
  f"()  -> {expand()}\n"
  f"()  -> {expand()}\n"
)
      \end{pyblock}

      \vspace{0.5em}
      This yields the output
      \vspace{0.5em}
      \printpythontex[verbatim][highlightlines={4}]
    \end{minipage}

    On the right-hand side, the change is that we modify \mintinline{python}{x} 
    before returning the new value.
    However, we notice in the output that something weird happens when we use 
    the default value of the parameter now:
    it seems like we actually update the default value every time we run the 
    function using the default value.

  \item[Generalisation] This brings us to the generalisation pattern.
    In the generalisation pattern we vary the non-critical aspects and keep the 
    critical aspect invariant.
    In this case, we just add more print statements to the example.
    (For brevity we don't repeat the definition of the 
    function~\mintinline{python}{expand}.)

    \begin{minipage}[t]{0.45\columnwidth}
      \begin{pyblock}[default1]
print(
  f"()  -> {expand()}\n"
  f"[3] -> {expand([3])}\n"
  f"()  -> {expand()}\n"
  f"[]  -> {expand([])}\n"
  f"[]  -> {expand([])}\n"
  f"()  -> {expand()}\n"
)
      \end{pyblock}
      \vspace{0.5em}
      This yields the output
      \vspace{0.5em}
      \printpythontex[verbatim]
    \end{minipage}
    \hfill
    \begin{minipage}[t]{0.45\columnwidth}
      \begin{pyblock}[default2]
print(
  f"()  -> {expand()}\n"
  f"[3] -> {expand([3])}\n"
  f"()  -> {expand()}\n"
  f"[]  -> {expand([])}\n"
  f"[]  -> {expand([])}\n"
  f"()  -> {expand()}\n"
)
      \end{pyblock}

      \vspace{0.5em}
      This yields the output
      \vspace{0.5em}
    \printpythontex[verbatim][highlightlines={1,3,6}]
    \end{minipage}

    We can see in the output that the default value in the argument keeps 
    expanding.
    In a sense, the generalisation pattern is the same as induction or the 
    scientific method.

    Now, we'll change the example, but keep this property invariant;
    or, phrased in terms of the scientific method, we try to falsify our 
    hypothesis.
    (We highlight the lines that keeps this property, \ie remains invariant in 
    the variation theoretic sense.)
    \begin{pyblock}[default1][highlightlines={10-12}]
class Person:
  def __init__(self, first, last):
    self.first = first
    self.last = last

  def __str__(self):
    return f"{self.first} {self.last}"

def create_person(first=None, last=None,
                  person_base=Person("Gina", "Jones")):
  if first: person_base.first = first
  if last: person_base.last = last
  return person_base

person_default = create_person()
print(person_default)
person_A = create_person("Ada")
print(person_A)
person_B = create_person("Beda")
print(person_B)
print(person_A)
print(person_default)
    \end{pyblock}
    \vspace{0.5em}
    This yields the output
    \vspace{0.5em}
    \printpythontex[verbatim][highlightlines={4-5}]

    We can see that the last three lines of the output is the same.
    The first is expected, but the last two (highlighted) indicates that they 
    refer to the same object; \ie that \mintinline{python}{person_A}, 
    \mintinline{python}{person_B} and \mintinline{python}{person_default} all 
    refer to the same object.

    This must be due to how Python is constructed.
    The default value seems to be constructed when the function is defined, not 
    when it's called (like in other languages, C++ for instance), and then 
    referenced (not copied) whenever the function is called without the 
    argument.
    Consider this example.

    \begin{pyblock}[default1][highlightlines=5]
class TraceClass:
  def __init__(self):
    print(f"{self} created")

def test_function(obj=TraceClass()):
  print(f"test_function called with obj = {obj}")

print("Test code begins")
test_function()
    \end{pyblock}
    \vspace{0.5em}
    This yields the output
    \vspace{0.5em}
    \printpythontex[verbatim][highlightlines={1}]

    We see that the print statement from the constructor is executed before the 
    test code is executed (highlighted line), supporting our hypothesis that 
    the default value is constructed when the function is defined, not the when 
    function is called, and then referenced throughout.

  \item[Fusion] Now we can fuse this back with our previous understanding of 
    default arguments, to see that it doesn't work with non-mutables like 
    integers.
    \begin{pyblock}[default1]
def increment(x=1):
  x += 1
  return x

print(f"(1) -> {increment(1)}")
print(f"(2) -> {increment(2)}")
print(f"()  -> {increment()}")
print(f"()  -> {increment()}")
print(f"()  -> {increment()}")
    \end{pyblock}
    \vspace{0.5em}
    This yields the output
    \vspace{0.5em}
    \printpythontex[verbatim][highlightlines={3-5}]

    And we can thus conclude that this phenomenon happens only with mutable 
    objects.
\end{description}

There are several things to note with this example.
First, note how the contrast pattern is designed to focus the your (the 
reader's) attention to the phenomenon at hand: that default values can change 
during execution.
Next, the generalisation pattern broadens our view of when this phenomenon 
happens, that the objects are referenced and reused.
Finally, the fusion pattern merges this back into our original view of default 
values, namely that they work as usual for non-mutable types (\eg integers).

Second thing to note, if you are a seasoned programmer, once you had that 
initial contrast pattern the remaining patterns (generalisation and fusion) 
probably resembles quite a lot what you would have tested yourself to make 
sense of this phenomenon---it would probably resemble how you would go about to 
\enquote{debug} this.

We actually tested this hypothesis on several colleagues who have been 
programming for many years, are well-versed in Python but didn't know of this 
phenomenon.
We gave them the contrast above and asked them to \enquote{debug} this 
behaviour and later explain it when they understood it.
(To record the data, we asked them to think aloud and recorded their screen and 
voice in a Zoom session.)

% XXX Test with more colleagues.
% Tested with Alexander.
The concrete examples that they tried varied from person to person, some tried 
many more examples than above, but the patterns of variation shown above were 
present.
Indeed, they couldn't explain the phenomenon until they had generated all the 
patterns above---covering both generalisation and fusion.

XXX add reference to NCOL that students taught according to variation theory, 
becomes better at creating these patterns for themselves.
In the context of programming, this should mean that if we teach them using 
variation theory, they would get better at debugging.

