\subsection{Data types}

When introduced to programming students will often encounter several 
different data types, for example strings, integers, arrays and 
dictionaries. As excepted when learning about several data types during a 
short time-span different misconceptions occur. 

One of the first data types that students encounter is strings, which can 
be seen as a simple type. However, \textcite{GuoMarkelZhang2020} found in 
their study that students often have a misconception about how one should 
use a string in a function call. When including a string as a parameter, for 
example \mintinline{python}{foo("Hello World")}, students instead write 
\mintinline{python}{foo(Hello World)}, where the misconception is about the syntax of 
declaring a string. 

XXX Add analysis on how to teach how to declare string-variables. Maybe 
this misconception should be in the section of functions and variables?

When using data types in CS1 the students are often required to use some 
kind of comparison between variables. This creates situations where the 
students try to compare different data types to each other, which shows 
that the students have misconceptions about the difference between data 
types and how they are initialised and then later used \parencite{
Kurvinen2016}. Another misconception that is common is what data type that 
is returned from terminal input or text-files. Both of these will return 
strings, but students often hold the misconception that when reading 
numbers from the terminal or files, they will automatically be converted to 
integers \parencite{GuoMarkelZhang2020}. This misconception will 
unfortunately not always raise an error, since many mathematical operators 
work for both strings and integers, and the misconception will instead lead 
to subtle errors later on in the programme. 

XXX Add analysis on how to introduce different data types so that the 
students understand the difference between them


 \textcite{Kurvinen2016} found in their studies that it is not intuitive 
for students that the indexation of an array starts at 0 instead of 1. 
They also noticed that students have a hard time figuring out how to loop 
through an array's elements by using the elements index. The same problem 
was found by \textcite{KumarVeerasamy2016} who found that students often 
are of by one index when looping through an array using index, causing 
index-error when trying to extract an element with an index larger then 
the length of the list. They also saw a tendency to use negative index-
numbers when unnecessary to extract elements from an array.

 XXX Insert analysis on how we can teach indexation of arrays