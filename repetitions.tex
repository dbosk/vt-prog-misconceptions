\subsection{Repetitions}

In CS1 students usually learn about repetitions, which includes for- and 
while-loops and in some cases recursion.

Loop constructions can be hard to trace and understand for novice students, 
for instance when a loop starts, ends and what is repeated and not repeated 
in the loop \parencite{Sekiya2013,KumarVeerasamy2016,Kaczmarczyk2010}. This 
was something \textcite{Sleeman1984} also realised when studying high 
school students writing and debugging loop-structures. A common 
misconception that the students had was that if the loop contained a print-
statement, the students thought that the only thing repeated inside the 
loop was the string they saw in the terminal. The difficulties students 
have in tracing the code linearly when entering a loop is according to 
\textcite{KumarVeerasamy2016} because of the lack of understanding the 
students have of the looping technique and the amount of cognitive skills 
the tracing takes.

XXX Add analysis on how we can help students trace loops and understanding 
how the loop-structure works.

When students define and use loops, \textcite{GuoMarkelZhang2020} found 
three common misconceptions that the students had about the loop-statement, 

\begin{enumerate}
    \item A misconception about which variable in a loop defined as 
      \mintinline{python}{for item in items}, that is supposed to be use to 
      extract different information.
    \item The misconception that \mintinline{python}{for i in 100} will iterate 
      a hundred times, even though Python require a specified range.
    \item When defining a while-loop, the misconception is that one can write 
      \mintinline{python}{while i <= 100} without initialising 
      \mintinline{python}{i} beforehand, and that \mintinline{python}{i} will 
      automatically increase with 1 inside the loop.
\end{enumerate}

Another difficult part of the loop technique is to understand how an if-
statement inside a loop is executed. \textcite{Sekiya2013} found in their 
studies that the combination of the two control structures created 
misconceptions. For instance the students thought that the variables in the 
conditional part of the loop-construction was control variables or the 
output from the loop. The students in the studies often got confused and 
started to misplace the different variables that are defined when writing 
an if-statement in a for-loop.

XXX Add analysis on how we can teach the combination of loops and 
conditionals in a way which will avoid misconceptions about the different 
variables used. 