% !TEX TS-program = pdflatex
% !TEX encoding = UTF-8 Unicode

% This is a simple template for a LaTeX document using the "article" class.
% See "book", "report", "letter" for other types of document.

\documentclass[twocolumn]{article}

%\documentclass[11pt]{article} % use larger type; default would be 10pt

\usepackage[utf8]{inputenc} % set input encoding (not needed with XeLaTeX)

\usepackage[english]{babel}

%%% Examples of Article customizations
% These packages are optional, depending whether you want the features they provide.
% See the LaTeX Companion or other references for full information.

%%% PAGE DIMENSIONS
\usepackage{geometry} % to change the page dimensions
\geometry{letterpaper, margin=1.3in} % or letterpaper (US) or a5paper or....
% \geometry{margin=2in} % for example, change the margins to 2 inches all round
% \geometry{landscape} % set up the page for landscape
%   read geometry.pdf for detailed page layout information

\usepackage{graphicx} % support the \includegraphics command and options

% \usepackage[parfill]{parskip} % Activate to begin paragraphs with an empty line rather than an indent

%%% PACKAGES
\usepackage{booktabs} % for much better looking tables
\usepackage{array} % for better arrays (eg matrices) in maths
\usepackage{paralist} % very flexible & customisable lists (eg. enumerate/itemize, etc.)
\usepackage{verbatim} % adds environment for commenting out blocks of text & for better verbatim
\usepackage{subfig} % make it possible to include more than one captioned figure/table in a single float

\usepackage{hyperref}
% These packages are all incorporated in the memoir class to one degree or another...

%%% HEADERS & FOOTERS
\usepackage{fancyhdr} % This should be set AFTER setting up the page geometry
\pagestyle{fancy} % options: empty , plain , fancy
\renewcommand{\headrulewidth}{0pt} % customise the layout...
\lhead{}\chead{}\rhead{}
\lfoot{}\cfoot{\thepage}\rfoot{}

%%% SECTION TITLE APPEARANCE
\usepackage{sectsty}
\allsectionsfont{\sffamily\mdseries\upshape} % (See the fntguide.pdf for font help)
% (This matches ConTeXt defaults)

%%% ToC (table of contents) APPEARANCE
\usepackage[nottoc,notlof,notlot]{tocbibind} % Put the bibliography in the ToC
\usepackage[titles,subfigure]{tocloft} % Alter the style of the Table of Contents
\renewcommand{\cftsecfont}{\rmfamily\mdseries\upshape}
\renewcommand{\cftsecpagefont}{\rmfamily\mdseries\upshape} % No bold!

%%% END Article customizations

%%% The "real" document content comes below...

\title{Teaching CS1}
\author{Celina Soori (at this moment)}
%\date{} % Activate to display a given date or no date (if empty),
         % otherwise the current date is printed 

\begin{document}
\maketitle

\section{Introduction}

Introduce and give som background on the subject and why we have chosen to analyze this. 

\subsection{Purpose}

The purpose of this article is to give an overview of the studies that has been conducted on this subject and analyze these with help of variation theory to understand how one is supposed to teach CS1 to help students grasp important concepts in learning programming. Another purpose with this article is to discuss what further studies we need to conduct to understand deeper what students difficulties there are and how the education can be designed to avoid these. 

\section{Method}

I'm thinking we need a method-section to describe how this research has been conducted. Something about the litterature collecting and so on. 

\section{Concepts}

Added this to explain different words we use in the text and how we interpreted them and use them. Maybe we dont need subsections for every word, but I started this way.

\subsection{Variation Theory}

Here I'm thinking we should describe what variation theory is so that we can use it in the analysis later on.

\subsection{Misconception}

Do we need to describe what we mean with misconception? Do we maybe want to use another word? 

\subsection{Computational thinking}

Maybe we want to include something about computational thinking in this article? 

\section{Important modules in teaching CS1}

This section is divided to reflect the different concepts that are teached during CS1. Each section will describe what students often are meant to learn and understand in that module, which is then followed with a summary of what different studies have found is difficult for students in that module. 

\subsection{Functions and variables}

\subsubsection{Role in the syllabus}

Do we want to have an introduction in every module to describe what the students are meant to learn when teaching CS1? Would be nice with som introduction, but maybe without a subsubsection, and only with one or two sentences. 

Example: Functions and variables are typically learned at the same time because they go hand in hand etc....

\subsubsection{Difficulties that can occure}

According to Yizhou Qian and James Lehman (2017) in their article \emph{Students’ Misconceptions and Other Difficulties in Introductory Programming: A Literature Review} students have difficulties understanding how variables and that the students usually make assumptions about variables that are wrong. For example one assumption is that variables can hold more then one value at the time. According to other studies sometimes students even think that variables can hold an entire algorithm and therefor see a variable as a function or equation. This will create problems when a student creates a variable in belief that the variable will change its value when the equation is supposed to change its value or at the time when the variable is used in the program (Kohn, 2017; Plass-Oude Bos, 2015). Kohn (2017) explains that this misconseption can be connected to how variable definitions are used in math. 

But it is not only the right side of the variable definition that students can have misconceptions about. The name of the variable has been misunderstood as having power of the value which it holds (Qian \& Lehman, 2017; Sleeman et al., 1984). For example if a student names one variable \emph{max} and another variable \emph{min}, the student might think that the variables will strictly only hold the maximum value and the minimum value througout the program. 

If we move on to the relationship between variables and functions we can see more conceptual misconseptions that students have, for example from where input and output arguments come from and go to (Ragonis \& Ben-Ari, 2005). The first difficulty is how students treat return-values. When a function is supposed to return a value some students miss the return value, expecting the function to return it by default (Kurvinen et al., 2016; Kumar Veerasamy et al., 2016). A student might also write a function which returns a value, but that value is not stored or being used later in the program (Altadmri \& Brown, 2015). Some students might also believe that a print-statement at the end of a function will act as a return statement (Qian \& Lehman, 2017). Another misconception that goes hand in hand with the assumption that a variable holds an equation and not a single value, is that if in the return statement the student returns an equation, the student believes that the return value will be that equation, not the value that the equation represents (Kohn, 2017). Some students also have difficulties returning the right value or variable from a function (Kumar Veerasamy et al., 2016). 

It is not only the return value that is difficult to grasp for students, the input arguments are also a difficult concept for some students. When calling a function it shows that the student have trouble using and understanding what arguments are meant to be used in the function call (Altadmri \& Brown, 2015). Ann E. Fleury (1991) researched the conceptual misconseptions students have when using parameters in functions, in her article \emph{Parameter passing: the rules the students construct}. What she found was that students had constructed their own rules for the using of global and local variables which were connected to the using of variables in functions. The first rule that a student had conducted was that when changing a local variable in a function, the variable was changed for the whole program. Another assumption made by the students was that if the local variable was not an argument in the function-call, the program would go back to where the function was called and search for it there. The assumptions the students had made about global variables was that if a function references to a global variable, it will create an error in the program because the global variable was not an argument in the function call. The students also believed that if a global variable was changed in the function body, the new value would not be reachable for the rest of the program if not returned by the function (Fleury, 1991).

The difference between variables in programming and variables in math is differences that some students does not grasp. If a student in a variable definition uses on the right side of the equal symbol a variable that is not defined, but the variable on the left side is already defined, they think that the computer will solve this as an equation (Plass-Oude Bos, 2015). This assumption by the students is also something discovered by Kohn (2017) when giving the students the definition \emph{x = x + 1}. If you look at this definition with a mathematical perspective you will see an unsolvable equation, which is also what some of the students saw. They did not see that the x to the left is the variable, and that the x to the right only holds a value. This definition is  easier to understand for a novice programmer according to Kohn (2017) when we instead write this definition as \emph{x += 1}. 


\subsection{Classes and objects}

\subsubsection{Role in the syllabus}

\subsubsection{Difficulties that can occure}

The concept of classes and objects in object-oriented languages is difficult and basic understanding of objects is something that many CS1 students lack (Kaczmarczyk et al., 2010). This is emphasized by Ragonis \& Ben-Ari (2005) who dive into this subject in their article \emph{A long-term investigation of the comprehension of OOP concepts by novices}. In their studie they noticed a number of diere misconceptions that the students had when learning about classes and objects, for instance that you can create an object from a method and that you can define a method that does not access any attributes. They also found that the students had a hard time to visualize the class as a template for a type of object, instead the students had the image of the class as a collection of objects and that the methods had the power to change, add and delete objects that are class-instances. 

Similar misconceptions has been characterized by Holland et al. (1997) in their article \emph{Avoiding object misconceptions} where they highlights the misconception that an object is a variable that can only hold one value or several values of the same type, a misconception they trace back to the first class examples that the students see. This misconception is not the only one they think is a symptome from the first classes the students see. The misconception that a class is strictly a data base is also a misconception that Holland et al. believes come from that the first classes the students write often are a good substitue to a data base and therefor shapes the student misconception. The last concept that Holland et al. discuss is the concept of storing the objects in the programme. Some students believe that the attributes of an object are the objects identifier, which leads to the misconception that there can not be two objects that have the same attributes, and therefor that one attribute of every object must be unique otherwise the programme will not be able to store it. The concept that every object has its own memory space and are stored separatly is something that is hard to grasp for some students (Holland et al., 1997; Ragonis \& Ben-Ari, 2005). 

\subsection{Data types}

\subsubsection{Role in the syllabus}

In this chapter I was thinking that we could combine lists, arrays, maybe dictionaries, strings, charachters and the comparison of different type of variables. Maybe I will found another type that can be included here. 

\subsubsection{Difficulties that can occure}

Index in lists: Programming misconceptions in an introductory level programming course exam av Einari Kurvinen, Niko Hellgren, Erkki Kaila, Mikko-Jussi Laakso, Tapio Salakoski

Comparision between different types: Programming misconceptions in an introductory level programming course exam av Einari Kurvinen, Niko Hellgren, Erkki Kaila, Mikko-Jussi Laakso, Tapio Salakoski

Doesnt understand how lists and arrays work: Veerasamy et al., 2016

\subsection{Repetitions}

\subsubsection{Role in the syllabus}

\subsubsection{Difficulties that can occure}

Variables in loops:Tracing quiz set to identify novices' programming misconceptions av Takayuki Sekiya, Kazunori Yamaguchi

Variables in if-statements that are in for-loops: Tracing quiz set to identify novices' programming misconceptions av Takayuki Sekiya, Kazunori Yamaguchi, Yizhou Qian and James Lehman (2017)

Understanding how to trace the execution-steps in a for loop: Veerasamy et al., 2016 Quote: \emph{Tracing loop execution needs cognitive skills, and students should have that ability to trace the code linearly. Our results reflected that a few students failed to trace the code linearly due to the lack of knowledge in “how the looping technique works.”}

Misconception about what is repeated in a loop, quote from Qian \& Lehman (originally from Sleeman et al 1984): \emph{ For example, if there are several lines inside a loop, and one of them is an output statement such as a print statement, students may think only the print statement is repeated by the loop because they only see the repeated output on the screen.}

\subsection{Problemsolving}

Here I want to have some articles about how math-problems will make it harder for students to solve the problem. Quote from Veerasamy et al \emph{This study analysis also explored that novices of programming struggled in writing code for math-related Questions 6 and 7 (refer Table C1). Nearly 66\% of students did not do well in the mathematical problem-based questions though explained and allowed to surf the Internet to seek for more details during the exam hours. A neo-Piagetian theory of cognitive development stated that students who are at the concrete operational stage struggle to write large programs with partial specifications, although they can write small programs from well-defined specifications (Teague et al., 2012).}

Also I would want to include difficulties students have when debugging the code and trying to find errors. Students often have a problem with tracing the code, something that is discussed in page 20 by Sleeman et al., 1984. 

Would also maybe like to mention how a lab instruction should be to help students get the right knowledge from the lab. Is discussed somewhere in Yizhou Qian and James Lehmans article I think.
\subsubsection{Role in the syllabus}

\subsubsection{Difficulties that can occure}

\subsection{Things that doesnt fit in anywhere else..}

In what order a program will be executed in: Programming misconceptions in an introductory level programming course exam by Einari Kurvinen, Niko Hellgren, Erkki Kaila, Mikko-Jussi Laakso, Tapio Salakoski

If-statements: If an if-statement is false the program will shut down, if true the program will keep on executing. Plass-Oude Bos, 2015. Also mentioned by Qian \& Lehman \emph{Conditionals are another difficult concept that leads to misconceptions (Green 1977; Sirkia 2012). Some students believe statements in both if and else blocks of a conditional expression will be executed (Sirkia 2012; Sleeman et al. 1986). Some students even mistakenly think that if the condition of an if-statement is false, the execution of the whole program stops (Sleeman et al. 1986)}

IDE: What IDE is best for CS1? What difficulties can occure when using different IDEs? Should we recommend one? Quote from Qian \& Lehman \emph{Although many other syntactic-level errors are reported in previous research (see Altadmri and Brown (2015), Hristova et al. (2003), and Sorva (2012)), we do not discuss them in depth here, because problems in syntactic knowledge are often easy to detect and fix. Perhaps that is why they are often noted as the most frequent mistakes novices make (Altadmri and Brown 2015; Jackson et al. 2005). A compiler or a modern integrated development environment (IDE) may be able to find them and then provide error messages or hints for correction.}


\section{Analysis}

The analyze will be performed with the help of variation theory. 

\section{Discussion}

Here we can discuss what we want to analyze and research further. We might also want to discuss how to design the education to avoid everything above.

\newpage

\begin{thebibliography}{9}

\bibitem{texbook}
Altadmri, A. Brown, N. C. C. (2015). 37 Million Compilations: Investigating Novice Programming Mistakes in Large-Scale Student Data. \emph{SIGCSE '15: Proceedings of the 46th ACM Technical Symposium on Computer Science Education}. February 2015. Pages 522-527.
\url{https://dl.acm.org/doi/abs/10.1145/2676723.2677258}

\bibitem{texbook}
Fleury, A. E. (1991). Parameter passing: The rules the students construct. \emph{SIGCSE’91:  In Proceedings of the 22nd SIGCSE Technical Symposium on Computer Science Education}. Pages 283-286.
\url{http://dx.doi.org/10.1145/107004.107066}

\bibitem{texbook}
Goldman, K. Gross, P. Heeren, C. Herman, G. Kaczmarczyk, L. Loui, M. C. Zilles, C. (2010). Identifying Important and Difficult Concepts in Introductory Computing Courses using a Delphi Process. \emph{SIGCSE '08: Proceedings of the 39th SIGCSE technical symposium on Computer science education} March 2008. Pages 256–260.
\url{https://dl.acm.org/doi/10.1145/1352135.1352226} 

\bibitem{texbook}
Holland, S. Griffiths, R. Woodman, M. (1997). Avoiding Object Misconceptions. \emph{ACM SIGCSE Bulletin} Volume 29, Issue 1, March 1997. Pages 131–134.
\url{https://dl.acm.org/doi/pdf/10.1145/268084.268132} 

\bibitem{texbook}
Kohn, T.  (2017). \emph{Variable Evaluation: an Exploration of Novice Programmers’ Understanding and Common Misconceptions}
\url{https://dl.acm.org/doi/abs/10.1145/3017680.3017724}

\bibitem{texbook}
Kurvinen, E. Hellgren, N. Kaila, E. Laakso, M. J. Salakoski, T. (2016). Programming misconceptions in an introductory level programming course exam. \emph{ITiCSE '16: Proceedings of the 2016 ACM Conference on Innovation and Technology in Computer Science Education}. July 2016. Pages 308–313. 
\url{https://doi.org/10.1145/2899415.2899447}

\bibitem{texbook}
Kumar Veerasamy, A. D’Souza, D. Laakso, M. J. (2016). Identifying Novice Student Programming Misconceptions and Errors From Summative Assessments. \emph{Journal of Educational Technology Systems.} Vol 45, Issue 1.  
\url{https://doi.org/10.1177/0047239515627263}

\bibitem{texbook}
Plass-Oude Bos, D.  (2015). Identifying and Addressing Common Programming Misconceptions with Variables (Part 1). \emph{University of Twente}.
\url{https://essay.utwente.nl/70455/1/Oude\%20Bos\%20Danny\%20-\%20S0021407\%20-\%20Afstudeerscriptie.pdf}

\bibitem{texbook}
Ragonis, N. Ben-Ari, M. (2005). A long-term investigation of the comprehension of OOP concepts by novices. \emph{ISSN: 0899-3408 (Print) 1744-5175 (Online) Journal homepage}. 15:3. 203-221. 
\url{https://doi.org/10.1080/08993400500224310}

\bibitem{texbook}
Sleeman, D. Putnam, R. T. Baxter, J. A. Kuspa, L. K. (1984). Pascal and High-School Students: A Study of Misconceptions. \emph{Technology Panel Study of Stanford and the Schools}. Occasional report \#009. 
\url{https://files.eric.ed.gov/fulltext/ED258552.pdf}

\bibitem{texbook}
Qian, Y. Lehman, J. (2017). Students’ Misconceptions and Other Difficulties in Introductory Programming: A Literature Review. \emph{ACM Trans. Comput. Educ.} 18, 1, Article 1 (October 2017).
\url{https://doi.org/10.1145/3077618} 



\end{thebibliography}

\end{document}
