\section{Introduction}


In higher education for technology there often exists introductory courses for programming, which often are named \emph{Computer Science 1} (CS1). When students are introduced to the world of programming some misconceptions are bound to happen, which we have seen when educating students at KTH. These obstacles that students meet when first learning to code made us interested in mapping which misconceptions that are common and understand the origin of these misconceptions. To map these misconceptions we rely on earlier studies that have been done in the area, combined with our own findings. With the help of this knowledge we hope to understand how one could develop the courses in introductory programming to avoid these misconceptions. In order to develop the course we are using variation theory, which we believe can be an effective tool in education. 


According to variation theory \parencite[Ch.~2]{NCOL}, each educational 
objective can be divided into different \emph{aspects}.
Consider the educational objective \enquote{the student should be able to use 
functions}.
One aspect of this particular educational objective is local and global scope 
of variables.
Another aspect is returning values from a function.
For a student to achieve the educational objective, she must be able to discern 
the different aspects of the educational objective.
Aspects that the student hasn't yet discerned are critical aspects.
One necessary condition for learning is that the student is introduced to a 
series of patterns of variation in the dimension of each critical aspect.
Misconceptions are examples of when a student has failed to discern (at least) 
one critical aspect.
This allows us to use misconceptions to inform our designs when designing 
teaching according to variation theory.

\subsection{Purpose}

The purpose of this study is to answer the following questions:
\begin{enumerate}
  \item What misconceptions in introductory programming has been identified by 
    earlier research?
  \item Based on these misconceptions, what aspects (in terms of variation 
    theory) of introductory programming can we identify?
  \item Where do we need further research?
\end{enumerate}